%________________________________________________________________________________________
% @brief    LaTeX2e Resume for Kamil K Wojcicki
% @author   Kamil K Wojcicki
% @url   http://linux.dsplabs.com.au/?p=54
% @date     Decemebr 2007
% @info     Based on Latex Resume Template by Chris Paciorek 
%        http://www.biostat.harvard.edu/~paciorek/
%________________________________________________________________________________________
\documentclass[margin,line]{resume}

\usepackage[latin1]{inputenc} %utf8
\usepackage[english]{babel}
\usepackage[T1]{fontenc}
\usepackage{graphicx,wrapfig}
\usepackage{url}
\usepackage[colorlinks=true, a4paper=true, pdfstartview=FitV,
linkcolor=blue, citecolor=blue, urlcolor=blue]{hyperref}
\pdfcompresslevel=9


\newcommand*{\doi}[1]{\href{http://dx.doi.org/#1}{doi: #1}}

\begin{document}
{\sc \Large Curriculum Vitae -- Khaled M. El-Wazan}
\begin{resume}
	\vspace{0.5cm}
	% \begin{wrapfigure}{R}{0.3\textwidth}
	%     \vspace{-1cm}
	%    \begin{center}
	%    \includegraphics[width=0.3\textwidth]{../Profile_Pictures/profile1.png}
	%    \end{center}
	%     \vspace{-1cm}
	% \end{wrapfigure}



	\section{\mysidestyle Personal\\Information}%\vspace{2mm}
	Khaled M. El-Wazan \\
	Research and Assistant Lecturer,\\
	Mathematics and Computer Science Department,\\
	Faculty of Science,  \\
	Alexandria University, \\
	Egypt. \\
	\emph{Tel:} +20 01228068942\\
	\emph{Email:} ~\href{mailto:khaled\_elwazan@ alex-sci.edu.eg}{khaled\_elwazan@alex-sci.edu.eg}\\
	\emph{github:} ~\href{https://github.com/KhaledElwazan}{https://github.com/KhaledElwazan}\\
	\emph{Orcid:} ~\href{https://orcid.org/0000-0002-8193-1602}{https://orcid.org/0000-0002-8193-1602}



	% I was born and raised in Alexandria where I have lived all my life, currently I am living alone.  I have always been very interested in computer science, and  by studying physics and mathematics, I had the opportunity of studying all combination of these three subjects. Currently, I work as a research and teaching assistant at mathematics and computer science department at Faculty of Science, which is quite ideal for me because I'm very passionate about learning new topics in computer science and communicating the acquired knowledge to students. 

	% In my free time,  I enjoy sports. I'm a regular weightlifter; I hit the gym about 5 days per week, and swimming classes and basketball training about two times a week. Beside sports, I love reading novels, especially horror and science fiction genres. My favorite authors are Stephen King, Joe Hill and Douglas Adams.


	\section{\mysidestyle Education}
	\textbf{PhD in Computer Science} (2021-Ongoing)
	\emph{Aims and Objectives:} Design and implementation of optimization strategies for the development of quantum circuits.

	\textbf{Master's degree in Mathematics: Computer Science (Quantum Computations) from the University of Alexandria, Egypt} (2015-2018).
	\emph{Thesis advisers}: Prof.  \href{https://scholar.google.com.eg/citations?user=CZz2XFIAAAAJ&hl=en}{Ahmed Younes} and Prof. \href{https://scholar.google.com.eg/citations?hl=en&user=YFeMsegAAAAJ&view_op=list_works&sortby=pubdate}{Salah B. Doma}.
	\emph{Thesis title}: \textit{Quantum Algorithms for Testing and Learning Boolean Functions}. Designing quantum algorithms for finding junta variables in a Boolean function provided as a black-box and learning certain Boolean functions.

	\textbf{Bachelor degree in computer science from the University of Alexandria, Egypt}
	(2008-2012).  \emph{Thesis adviser}: Dr. \href{https://scholar.google.com.eg/citations?hl=en&user=G9tQkdIAAAAJ&view_op=list_works&sortby=pubdate}{Ashraf Elsayed}.
	Thesis title: \textit{Intelligent E-Learning Systems via Means of Computer Vision.}


	\section{\mysidestyle Current Affiliations}\vspace{1mm}
	\begin{description}
		% \item[Academy of Scientific Research and Technology(ASRT)]
		\item[Department of Mathematics and Computer Science, Faculty of Science]
	\end{description}


	\section{\mysidestyle Job experiences}\vspace{1mm}
	\begin{description}


		\item[2020 March $\rightarrow$ 2022] Was awarded a funding from the Academy of Scientific Research and Technology (ASRT) for our research project on quantum circuit development and optimization. Our efforts were recognized in the form of publication in a Q2 journal, available at \doi{10.3390/sym13101842}

		\item[2021 Match $\rightarrow$ Present] I have been employed as an assistant lecturer at the Faculty of Science (SIM Program), Alexandria National University in Egypt. My responsibilities include serving as a tutor and mentor to students, distributing assignments, grading papers, meeting with students during office hours, assisting professors in developing course plans, teaching undergraduate courses, and maintaining records of attendance and student performance.

			\emph{Courses}: Introduction to Programming, Data Structures and Algorithms, Object Oriented Programming.

		\item[2013 July $\rightarrow$ Present]  I have been employed as a research and assistant lecturer at the Faculty of Science, Alexandria University in Egypt. My responsibilities include conducting academic research, serving as a tutor and mentor to students, distributing assignments, grading papers, meeting with students during office hours, assisting professors in developing course plans, teaching undergraduate courses, and maintaining records of attendance and student performance.

			\emph{Courses}: Discrete Structures, Structured Programming using C, Object Oriented Programming using Java, Data Structures using C++ , Algorithm Analysis and Design, Computational Complexity, Computer Architecture, Operating Systems Concepts and Design, Compiler Design and Construction, Information Storage and Retrieval, Distributed Systems, Theory of Computation and Multimedia Systems and Applications.


		\item[2017 July $\rightarrow$ December 2020 ]  Joined a research group at Alexandria University, \href{http://www.sci.p.alexu.edu.eg/~aleqcg/index.html}{Alexandria Quantum Computing Group}.




		\item[2015 February $\rightarrow$ 2018 July]  I was employed as a teaching assistant at the Faculty of Engineering, Alexandria University in Egypt. My responsibilities included serving as a tutor and mentor to students, distributing assignments, grading papers, meeting with students during office hours, assisting professors in developing course plans, teaching undergraduate courses, and maintaining records of attendance and student performance.

			\emph{Courses}: Multimedia Systems and Applications, Switching Theory.

	\end{description}



	\section{\mysidestyle Extracurricular activities}\vspace{1mm}
	\begin{description}
		\item[2014 October $\rightarrow$ 2016 August] Coached Faculty of Science ACM teams to the final rounds at Egyptian Collegiate Programming Contest (ACM-ECPC).

	\end{description}

	\section{\mysidestyle Grants}
	\begin{description}
		\item[2020 March $\rightarrow$ 2022] Worked on a funded research project by the \textit{Academy of Scientific Research and Technology (ASRT), Grand number 6614}. Our work was published in a Q2 journal \doi{10.3390/sym13101842}.
	\end{description}


	\section{ \mysidestyle Published papers}

	Mohamed Osman, and Khaled El-Wazan. "Efficient Designs of Quantum Adder/Subtractor Using Universal Reversible Gate on IBM Q." Symmetry 13.10 (2021) \doi{10.3390/sym13101842}.


	Khaled El-Wazan, "A Quantum Algorithm for Finding Common Matches between Databases with Reliable Behavior," Quantum Information Review, vol. 6, pp. 1-6 (2019) \doi{10.18576/qir/060101}.


	\section{\mysidestyle Reports}

	Khaled El-Wazan, "Measuring Hamming Distance between Boolean Functions via Entanglement Measure ,"  pp. 1--12, 2019.
		[Online]. Available: \href{http://arxiv.org/abs/1903.04762}{1903.04762}.



	Khaled El-Wazan, Ahmed Younes and Salah B. Doma, "A Quantum Algorithm for Testing Junta Variables and Learning Boolean Functions via Entanglement Measure,"  pp. 1--15, 2017.
		[Online]. Available: \href{http://arxiv.org/abs/1710.10495}{1710.10495}.



	\section{ \mysidestyle Conference contributions}


	Khaled El-Wazan, Ahmed Younes and Salah B. Doma, "A Quantum Algorithm for Testing Juntas in Boolean Functions," \textit{One day conference of Quantum Computer and Quantum Information}, Faculty of Science, Alexandria University, Egypt, 2016. [Online]. Available: \href{http://arxiv.org/abs/1701.02143}{1701.02143}.


	\section{\mysidestyle Training Courses and Certificates}
	\begin{itemize}

		\item Ethical Conduct and Code of Ethics (IBCT-FLDC).
		\item Global Databases Usage (IBCT-FLDC).
		\item Research Methods (IBCT-FLDC).
		\item Presentation Skills (IBCT-FLDC).
		\item Rights and Duties of University Staff Assistants (IBCT-FLDC).
		\item Experimental Design and Statistical Analysis Systems (IBCT-FLDC).
		\item Japanese Language Proficiency (JF).

	\end{itemize}




	\section{\mysidestyle Computer skills}\vspace{1mm}
	\begin{description}
		\item[Operating systems] Advanced experience with various flavors of Linux, including Arch, Ubuntu, Mint and Debian. Experienced with Microsoft Windows and Mac OSX to some extent.
		\item[Servers and databases] Proficient in Apache2, NoSQL, GlassFish, MySQL, ESXI Hypervisor, VMWare vSphere, Apache Hadoop and Apache Pig.
		\item[Programming, scripting and markup languages] Proficient in Python, Bash (used daily) and JavaScript. Familiar with \LaTeX, PHP, HTML, Octave, C, C++, C Sharp, Java SE and EE, Android SDK, Flutter SDK, Nodejs, FORTRAN and Pig Latin.
	\end{description}


	\section{\mysidestyle Language skills}
	My mother tongue is Arabic and most of my writing, including content related to computers on my blog  \href{https://qaqcblog.quora.com/}{Quantum Algorithms and Quantum Computations}", as well as in my academic and scientific work, is done in English.
	\textbf{Arabic}: Native speaker.
	\textbf{English}: Intermediate level proficiency.
	\textbf{Japanese}: Beginner level (JLPT N5 certified).

\end{resume}
\end{document}
