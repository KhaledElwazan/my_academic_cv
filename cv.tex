%________________________________________________________________________________________
% @brief    LaTeX2e Resume for Kamil K Wojcicki
% @author   Kamil K Wojcicki
% @url   http://linux.dsplabs.com.au/?p=54
% @date     Decemebr 2007
% @info     Based on Latex Resume Template by Chris Paciorek 
%        http://www.biostat.harvard.edu/~paciorek/
%________________________________________________________________________________________
\documentclass[margin,line,a4paper]{resume}

\usepackage[latin1]{inputenc} %utf8
\usepackage[english,danish]{babel}
\usepackage[T1]{fontenc}
\usepackage{graphicx,wrapfig}
\usepackage{url}
\usepackage[colorlinks=true, a4paper=true, pdfstartview=FitV,
linkcolor=blue, citecolor=blue, urlcolor=blue]{hyperref}
\pdfcompresslevel=9


\newcommand*{\doi}[1]{\href{http://dx.doi.org/#1}{doi: #1}}

\begin{document}
{\sc \Large Curriculum Vitae -- Khaled M. El-Wazan}
\begin{resume}
    \vspace{0.5cm}
    % \begin{wrapfigure}{R}{0.3\textwidth}
    %     \vspace{-1cm}
    %    \begin{center}
    %    \includegraphics[width=0.3\textwidth]{../Profile_Pictures/profile1.png}
    %    \end{center}
    %     \vspace{-1cm}
    % \end{wrapfigure}
    


    \section{\mysidestyle Personal\\Information}%\vspace{2mm}
    Khaled M. El-Wazan \\
    Research and Teaching Assistant,\\
    Mathematics and Computer Science Department,\\
    Faculty of Science,  \\ 
    Alexandria University, \\ 
    Egypt. \\ 
    \emph{Tel:} +20 01228068942\\
    \emph{Email:} \\~\href{mailto:khaled\_elwazan@ alex-sci.edu.eg}{khaled\_elwazan@alex-sci.edu.eg}\\
    \emph{github:} \\~\href{https://github.com/KhaledElwazan}{https://github.com/KhaledElwazan}\\
    
    
% \emph{Skype:}\\~\href{mailto:khaled\_elwazan@ alex-sci.edu.eg}{khaled\_elwazan@alex-sci.edu.eg}    
    
    
    I was born and raised in Alexandria where I have lived all my life, currently I am living alone.  I have always been very interested in computer science, and  by studying physics and mathematics, I had the opportunity of studying all combination of these three subjects. Currently, I work as a research and teaching assistant at mathematics and computer science department at Faculty of Science, which is quite ideal for me because I'm very passionate about learning new topics in computer science and communicating the acquired knowledge to students. 
%    A typical day of work could either be spent doing wavelet extraction on seismic data
%    or writing a munin plugin to various services on the servers. 
    
        In my free time,  I enjoy sports. I'm a regular weightlifter; I hit the gym about 5 days per week, and swimming classes and basketball training about two times a week. Beside sports, I love reading novels, especially horror and science fiction genres. My favorite authors are Stephen King, Joe Hill and Douglas Adams.

        
    \section{\mysidestyle Education} 
    \textbf{PhD in Computer Science} (2021-Ongoing)
     \emph{Aims and Objectives:} Design and implementation of optimization strategies for the development of quantum circuits.    

    \textbf{Master's degree in Mathematics: Computer Science (Quantum Computations) from the University of Alexandria, Egypt} (2015-2018). 
    \emph{Thesis advisers}: Prof.  \href{https://scholar.google.com.eg/citations?user=CZz2XFIAAAAJ&hl=en}{Ahmed Younes} and Prof. \href{https://scholar.google.com.eg/citations?hl=en&user=YFeMsegAAAAJ&view_op=list_works&sortby=pubdate}{Salah B. Doma}. 
     \emph{Thesis title}: \textit{Quantum Algorithms for Testing and Learning Boolean Functions}. Designing quantum algorithms for finding junta variables in a Boolean function provided as a black-box and learning certain Boolean functions.

    \textbf{Bachelor degree in computer science from the University of Alexandria, Egypt}
    (2008-2012).  \emph{Thesis adviser}: Dr. \href{https://scholar.google.com.eg/citations?hl=en&user=G9tQkdIAAAAJ&view_op=list_works&sortby=pubdate}{Ashraf Elsayed}.
    Thesis title: \textit{Intelligent E-Learning Systems via Means of Computer Vision.}

%    \textbf{Rysensteens Gymnasium} (1998-2001) High school.  I attended a special mathematical/physical line. Located in Copenhagen.
%    \textbf{Den Classenske Legatskole} (1989-1998) Municipal School.
%    

\section{\mysidestyle Current Affiliations}\vspace{1mm}
\begin{description}
    \item[Academy of Scientific Research and Technology(ASRT)]
    \item[Department of Mathematics and Computer Science, Faculty of Science] 
\end{description}


\section{\mysidestyle Job experiences}\vspace{1mm}
\begin{description}
    

    \item[2020 March $\rightarrow$ Present] Working on a funded research project by the ASRT.

    
    \item[2013 July $\rightarrow$ Present] Employed as a research and assistant lecturer at Faculty of Science, Alexandria University, Egypt. 
    
    \emph{Tasks}: includes academic research, tutor/mentor students, hand out assignments and grade papers, meet with students during office hours, help professors develop course plans, teach undergraduate courses and take attendance and record responses.
       
    \emph{Courses}: Discrete Structures, Structured Programming using C, Object Oriented Programming using Java, Data Structures using C++ , Algorithm Analysis and Design, Computational Complexity, Computer Architecture, Operating Systems Concepts and Design, Compiler Design and Construction, Information Storage and Retrieval, Distributed Systems, Theory of Computation and Multimedia Systems and Applications.
    
        
	\item[2017 July $\rightarrow$ December 2020 ]  Joined a research group at Alexandria University, \href{http://www.sci.p.alexu.edu.eg/~aleqcg/index.html}{Alexandria Quantum Computing Group}.
    

        

	\item[2015 February $\rightarrow$ 2018 July] Employed as a teaching assistant at Faculty of Engineering, Alexandria University, Egypt. 
	
	\emph{Tasks}: includes tutor/mentor students, hand out assignments and grade papers, meet with students during office hours, help professors develop course plans, teach undergraduate courses and take attendance and record responses.        
    
	\emph{Courses}: Multimedia Systems and Applications, Switching Theory.
    
 
    


%    \item[2009 January (Thomas Jansson IT)] Constructed web frontend for 
%    the ``Shallow Water Model'' for use in teaching at the geophysical
%    department of University of Copenhagen. Referee: Eigil Kaas
%    (\href{mailto:kaas@gfy.ku.dk}{kaas@gfy.ku.dk}) and Aksel Walløe Hansen
%    (\href{mailto:awh@gfy.ku.dk}{awh@gfy.ku.dk}).
%    \item[2008 August $\rightarrow$ October (Thomas Jansson IT)] Gave a one-day course 
%    in the use of the content management system Drupal for DTM International
%    A/S. Subsequently employed as a consultant. 
%    \item[2008 July (Thomas Jansson IT)] Building website for "First Workshop on Satellite Imaging
%    of the Arctic", see \url{www.gfy.ku.dk/~awh/satellite-imaging/}. 
%    \item[2008 June (Thomas Jansson IT)] Constructed web frontend for 
%     the ``Simple Meridional Energy Balance Model'' for use in teaching at the geophysical
%    department of University of Copenhagen. Referee: Eigil Kaas
%    (\href{mailto:kaas@gfy.ku.dk}{kaas@gfy.ku.dk}). see
%    \url{http://gfy.ku.dk/~kaas/onedmodel/run.php}. 
%    \item[2008 January $\rightarrow$ October (Thomas Jansson IT) ] Further development of the python-
%    based graphical user interface, pyGravsoft. This time the program was
%    tested in Malaysia and included user surveys. 
%    \item[2007 June (Thomas Jansson IT) ] A PHP/MySQL based web page for the magazine Kvant, see
%    \url{www.kvant.dk}. The new site has a searchable index of every
%    article in Kvant. 
%    \item[2007 May (Thomas Jansson IT) ] Building a python based graphical user interface to a text
%    based gravimetric program called Gravsoft, see
%    \url{www.gfy.ku.dk/~cct/}. Referee: Professor Carl Christian Tscherning
%    (\href{mailto:cct@gfy.ku.dk}{cct@gfy.ku.dk}).
%    \item[2007 May (Thomas Jansson IT)] Building a Xoops (CMS) based web page for LJ Ejendomme, see
%    \url{www.lj-ejendomme.dk}.
%    \item[2006 August $\rightarrow$ 2009 Feburary (Thomas Jansson IT)] Unix system administrator at the Geological
%    Institute, University of Copenhagen. Work tasks: Normal system
%    administration of SUN and Linux servers as well as setting up a small Linux
%    cluster for sea modeling. Referee: Professor Hans Thybo,
%    \href{mailto:thybo@geo.ku.dk}{thybo@geo.ku.dk}. 
%    \item[2006 July (Thomas Jansson IT)] Building and setup of a Linux based file and applications
%    server for Utopiarejser.  
%    \item[2006 March (Thomas Jansson IT)] Building a web page for ``Copenhagen Global Change
%    Initiative'', see \url{www.klima.nbi.dk}. 
%    \item[2006 January $\rightarrow$ ] Started a company:\textit{Thomas Jansson IT}.
%    I am doing consulting jobs as a programmer, system administrator, server
%    building and web design. In this connection I write IT articles on my blog 
%    \url{www.tjansson.dk}.
%    \item[2005 January $\rightarrow$ 2005 July ] Substitute math and physics teacher at Bjørns
%    international school (elementary school). 
\end{description}

%
%\section{\mysidestyle Courses taken (pre-master)}\vspace{1mm}
%\begin{description}
%    \item[]
%\end{description}


\section{\mysidestyle Extracurricular activities}\vspace{1mm}
    \begin{description}
    \item[2014 October $\rightarrow$ 2016 August] Coached Faculty of Science ACM teams to the final rounds at Egyptian Collegiate Programming Contest (ACM-ECPC).
    
    
    
%    \item[2007 December $\rightarrow$ ] Editor at Kvant. 
%    \item[2006 August $\rightarrow$ December ] Studies at UNIS, Svalbard. I
%    studied oceanography and remote sensing for one semester at Svalbard,
%    78$^\circ$ N.
%    \item[2001 December $\rightarrow$ 2008 June ] Editor at \emph{Gamma}.
%    \emph{Gamma} is student-operated magazine sponsored by the Niels Bohr
%    Institute. 3000 copies four times a year. Functioned as editor in chief
%    in several periods. 
%    \item[2007 February $\rightarrow$ 2008 September ] Board member in the
%    geophysical student union.
    \end{description}



\section{ \mysidestyle Published papers}

Mohamed Osman, and Khaled El-Wazan. "Efficient Designs of Quantum Adder/Subtractor Using Universal Reversible Gate on IBM Q." Symmetry 13.10 (2021) \doi{10.3390/sym13101842}.
    
    
Khaled El-Wazan, "A Quantum Algorithm for Finding Common Matches between Databases with Reliable Behavior," Quantum Information Review, vol. 6, pp. 1-6 (2019) \doi{10.18576/qir/060101}.
    % [Online]. Available: \href{http://arxiv.org/abs/1704.01204}{1704.01204}.


\section{\mysidestyle Reports}

Khaled El-Wazan, "Measuring Hamming Distance between Boolean Functions via Entanglement Measure ,"  pp. 1--12, 2019. 
[Online]. Available: \href{http://arxiv.org/abs/1903.04762}{1903.04762}.



Khaled El-Wazan, Ahmed Younes and Salah B. Doma, "A Quantum Algorithm for Testing Junta Variables and Learning Boolean Functions via Entanglement Measure,"  pp. 1--15, 2017. 
[Online]. Available: \href{http://arxiv.org/abs/1710.10495}{1710.10495}.



%Khaled El-Wazan, "A Quantum Algorithm for Finding Common Matches Between Databases with Reliable Behavior," pp. 1--10, 2017. [Online]. Available: \href{http://arxiv.org/abs/1704.01204}{1704.01204}.
%
%Khaled El-Wazan, Ahmed Younes and Salah B. Doma, "A Quantum Algorithm for Testing Juntas in Boolean Functions," pp. 1--13, 2017. 
%[Online]. Available: \href{http://arxiv.org/abs/1701.02143}{1701.02143}.
%% 


\section{ \mysidestyle Conference contributions}


Khaled El-Wazan, Ahmed Younes and Salah B. Doma, "A Quantum Algorithm for Testing Juntas in Boolean Functions," \textit{One day conference of Quantum Computer and Quantum Information}, Faculty of Science, Alexandria University, Egypt, 2016. [Online]. Available: \href{http://arxiv.org/abs/1701.02143}{1701.02143}.

%    R. Forsberg. T. R. N. Jansson, J. E. Nielsen, C. C. Tscherning,
%    \emph{Development of a Python interface to the GRAVSOFT gravity field
%    programs.}, IAG 2009 - Geodesy for Planet Earth, Buenos Aires
%    (September 2009).  
%
%    D. G. Bennett, H. Changela, N. Dalcher, C. L. Goldmann, M. Heger, T.
%    Hiriart, T. R. N. Jansson, S. Kern, K. Motamedi, M. Petitat, G.
%    Sangiovanni, J. Spurmann, A. Stiegler, M. Unterberger, E. Vigren.
%    \emph{I. T. -- R. O. C. K. S. Comet Nuclei Sample Return Mission}, International
%    Astronautical Congress September 2008, Glassgow. 
%
%    Tomas Bohr, Pascal Hersen, Thomas R. N. Jansson, Martin P. Haspang and K. H.
%    Jensen, \emph{Polygons on a Rotating Fluid Surface}, The 6th Euromech Fluid Mechanics
%    Conference, Stockholm (June 2006)
%
%    T.R.N. Jansson, M.P. Haspang, K.H. Jensen, P. Hersen \& T. Bohr,
%    \emph{Polygons on a
%    Rotating Fluid Surface}, Second International Symposium on Instability and
%    Bifurcations in Fluid Dynamics, Technical University of Denmark (August 2006)


\section{\mysidestyle Training Courses and Certificates}
\begin{itemize}

\item Ethical Conduct and Code of Ethics (IBCT-FLDC).
\item Global Databases Usage (IBCT-FLDC).
\item Research Methods (IBCT-FLDC).
\item Presentation Skills (IBCT-FLDC).
\item Rights and Duties of University Staff Assistants (IBCT-FLDC).
\item Experimental Design and Statistical Analysis Systems (IBCT-FLDC).
\item Japanese Language Proficiency (JF).

\end{itemize}



%\section{ \mysidestyle Presentations}
%The following presentations was can be found on \url{www.tjansson.dk}. All the
%presentation was held at the University of Copenhagen. 
%\begin{itemize}
%    \item \emph{Receiver function modellering}, Geofysikdag held by ``Dansk
%    Geofysisk Forening'',  11 April 2008. 
%    \item \emph{Dmpning af seismiske blger}, Hovedfagskollokvium, 21 December
%    2007. 
%    \item \emph{Et bachelorprojekt om roterende vand}, inspirational talk for
%    new bachelor students, 9 November 2007.
%\end{itemize}

%\section{\mysidestyle Selected popular science articles}
%As mentioned earlier I was the editor at Gamma for 7 years and later started at
%Kvant. During this period I have written 24 small articles and news stories on
%physics and technology. The articles can be found on the pages:
%\url{www.kvant.dk} and \url{www.gamma.nbi.dk}. Where no other authors are
%stated I am the author.
%
%\begin{itemize}
%    \item \emph{Review: ``Kvantespring i det 20. rhundrede''}, Gamma,
%    fall, 2008.
%    \item \emph{Review: ``Insultingly stupid movie physics''}, Kvant 3,
%    2008.
%    \item \emph{Eksperiment med flydende metaller relateret til jordens
%    magnetfelt}, Gamma 145, 2007.
%    \item \emph{Ru vingeoverflade kan spare brndsel}, Gamma 141, 2006.
%    \item Kre H. Jensen og Thomas R. N. Jansson, \emph{Open source programmer til
%    videnskabelig brug}, Gamma 140, 2005.
%    \item Alexandru Nicolin, Thomas R. N. Jansson og Andreas Lemark,
%    \emph{Interview med Nobelpristager David J. Gross}, Gamma 138, Maj 2005.
%    \item \emph{Review: ``Fra superstrenge til stjerner''},
%    Gamma 132, 2003.
%\end{itemize}

    
    

\section{\mysidestyle Computer skills}\vspace{1mm}
\begin{description}
\item[Operating systems] Advanced experience with most flavors of Linux, \textit{e.g.} Ubuntu, Mint and Debian. Experienced with Microsoft Windows and Mac OSX to some extent.
    \item[Servers and databases] Apache2, NoSQL, GlassFish, MySQL, ESXI Hypervisor, VMWare vSphere, Apache Hadoop and Apache Pig.


    \item[Programming, scripting and markup languages] Python, Bash (daily) and JavaScript. \LaTeX, PHP, HTML, Octave (Often). C, C++, C sharp, Java SE and EE, Android SDK, Flutter SDK, Nodejs, FORTRAN and Pig Latin.
%    \item[Courses] Attended 5 days NetApp course, 5 days RHCE Rapid Track Course.  
%    \item[Certifications] Red Hat Certified Technician.  
%    \item[Open source projetcs] Co-author and owner of the python based open
%    source project Sinthgunt.  An easy python/GTK frontend to ffmpeg using more
%    than 100 pre-configured conversion settings. Included in the repositories
%    of various Linux distributions.\\ \url{http://code.google.com/p/sinthgunt/}
\end{description}
    

\section{\mysidestyle Language skills}
    My mother tongue is Arabic, but almost everything I write is in English, both in connection to computers in general on my blog \href{https://qaqcblog.quora.com/}{Quantum Algorithms and Quantum Computations} and in academic and scientific work. \textbf{Arabic}: Native tongue. \textbf{English}: Intermediate. \textbf{Japanese}: Beginner level (JLPT N5 Certified).
   


% \section{\mysidestyle Academic Referees}
% \begin{description}


% \item[ Prof. Dr. Salah B. Doma]


% Tel: +20 01223991249\\
% E-mail: sbdoma@yahoo.com\\
% Webpage: \href{https://www.researchgate.net/profile/Salah\_Doma}{https://www.researchgate.net/profile/Salah\_Doma}



% \item[ Prof. Dr. Ahmed Younes]


% Tel: +20 01001368289\\
% E-mail: ayounes@alexu.edu.eg\\
% Webpage: \href{https://ayounes.page.tl/}{https://ayounes.page.tl/}




% \item[ Dr. Mohamed Abd-El-Rahman]




% Tel: +20 01111883060\\
% E-mail: m.abdou@pua.edu.eg\\
% Webpage: \href{http://www.pua.edu.eg/UserFiles/file/Engineering/cvs/2017/Dr.\%20Mohamed\%20Abdel\%20Rahman.doc}{C.V.}



% \end{description}
\end{resume}
\end{document}
